\chapter{\IfLanguageName{dutch}{Stand van zaken}{State of the art}}
\label{ch:stand-van-zaken}

% Tip: Begin elk hoofdstuk met een paragraaf inleiding die beschrijft hoe
% dit hoofdstuk past binnen het geheel van de bachelorproef. Geef in het
% bijzonder aan wat de link is met het vorige en volgende hoofdstuk.

% Pas na deze inleidende paragraaf komt de eerste sectiehoofding.

Dit hoofdstuk bevat je literatuurstudie. De inhoud gaat verder op de inleiding, maar zal het onderwerp van de bachelorproef *diepgaand* uitspitten. De bedoeling is dat de lezer na lezing van dit hoofdstuk helemaal op de hoogte is van de huidige stand van zaken (state-of-the-art) in het onderzoeksdomein. Iemand die niet vertrouwd is met het onderwerp, weet nu voldoende om de rest van het verhaal te kunnen volgen, zonder dat die er nog andere informatie moet over opzoeken \autocite{Pollefliet2011}.

\section{Microservices}
\subsection{Definitie}
Volgens ~\cite{Mauersberger2017} is de definitie van microservices "A method of developing software applications as a suite of independently deployable, small, modular services in which each service runs a unique process and communicates through a well-defined, lightweight mechanism to serve a business goals.". Als je de definitie onderverdeeld, zie je drie onderdelen. Het eerste onderdeel vertelt hoe een microservice in elkaar zit. Het is een onafhankelijke, kleine, modulaire services. Deze service communiceert op een eenvoudige manier. Dit is een tweede eigenschap van een microservice. De derde eigenschap is een microservice wordt gemaakt in functie van een requirement uit de business. 
~\cite{series2018} vindt het doel van microservices: de problemen van monolith verhelpen met microservices. De vorige definite legde uit wat microservices zijn. Dit artikel zegt waar men microservices kan plaatsen. Er is dus één groot framework. Daar zitten meerdere onafhankelijke services in.
Een andere definitie voor microservices is "A software architecting pattern that allows software to be developed into relatively small, distinct components. Each of the components is abstracted by an API(s) and provides a distinct subset of the functionality of the entire application". Ook hier zien we weer het puntje passeren dat een microservice een klein componentje is van een groter geheel. ~\cite{series2018} benaderukt die eigenschap van een microservices heel hard. 
\subsection{Het belang van microservices}
Tegenwoordig werken mensen vaak Agile. Dat vraagt na twee weken een afgewerkt stukje software. Bij een monolith kan het aanpassen van een deeltje, veel werk vragen. Microservices spelen daar gemakkelijk op in. Die technologie legt niet heel het framework plat als er deeltjes moeten bij gecodeerd worden. Microservices kunnen sneller inspelen op de Agile analyse methode.
~\cite{series2018} haalt aan dat microservices van belang zijn bij het scalen van software.
\subsection{Algemene aanpak om microservices te implementeren}
Het interessante artikel van ~\cite{Benetis2016} over een 6-stappen plan om microservices te implementeren.
De eerste stap is het bepalen van de business requirements die de microservices zal bedienen. Dit is ook een van de belangrijkste redenen om microservices te gaan gebruiken. Het voldoen aan de business requirements. 
\subsection{De voordelen en nadelen van microserivces}
In het artikel van ~\cite{series2018} wordt er veel lofzang gedaan over microservices. Het gebruik van microservices zou ervoor zorgen dat de architectuur flexibeler wordt. Er kunnen microservices hergebruikt worden. Dankzij microservices is het hermodeleren, implementeren van nieuwe technologieën, ... 
Kleinere deeltjes zijn gemakkelijker te documenteren. De snelheid van microservices zijn een groot pluspunt.
\subsection{Voorbeelden}
In dit deeltje zal je meer te weten komen over hoe grote technologische bedrijven microservices toepassen en hoe ze naar deze technologie zijn overgeschakeld.
Een term die hier vaak zal gebruikt worden, is een monolith. Dit is de tegenhanger van microservices. Sommige zweren bij monolith en anderen hebben gouden woorden voor microservices. De beslissing om één van de twee technieken te kiezen, ligt bij wat je precies nodig hebt. Wat de business nodig heeft. 
\subsubsection{Amazon}
~\cite{Mauersberger2017} gaf een voorbeeld waarom Amazon overstapte. Zoals veel grote bedrijven is Amazon begonnen met een grote monolith. Een van de nadelen die Amazon ondervond aan deze technologie is de moeilijkheid van het inschatten van de zwaarte op de bandbreedt. Daardoor verloor Amazon veel geld en was er nood aan herstructurering.
\subsubsection{Apple}
\subsubsection{Facebook}
\subsubsection{Netflix}

\section{Order-to-cash proces in SAP}
\subsection{Definite}
De definitie van een order-to-cash proces.
\subsection{Technologie}
\subsubsection{Onderdelen van een order-to-cash proces}
\subsubsection{Wat biedt SAP zelf aan voor microservices}
\subsection{Een order-to-cash proces vanuit de business}
\subsection{Het proces afstemmen met de business}

\section{Requirements van de business}
Je verwijst bij elke bewering die je doet, vakterm die je introduceert, enz. naar je bronnen. In \LaTeX{} kan dat met het commando \texttt{$\backslash${textcite\{\}}} of \texttt{$\backslash${autocite\{\}}}. Als argument van het commando geef je de ``sleutel'' van een ``record'' in een bibliografische databank in het Bib\LaTeX{}-formaat (een tekstbestand). Als je expliciet naar de auteur verwijst in de zin, gebruik je \texttt{$\backslash${}textcite\{\}}.
Soms wil je de auteur niet expliciet vernoemen, dan gebruik je \texttt{$\backslash${}autocite\{\}}. In de volgende paragraaf een voorbeeld van elk.

\textcite{Knuth1998} schreef een van de standaardwerken over sorteer- en zoekalgoritmen. Experten zijn het erover eens dat cloud computing een interessante opportuniteit vormen, zowel voor gebruikers als voor dienstverleners op vlak van informatietechnologie~\autocite{Creeger2009}.

\lipsum[7-20]
