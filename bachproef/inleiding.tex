%%=============================================================================
%% Inleiding
%%=============================================================================

\chapter{\IfLanguageName{dutch}{Inleiding}{Introduction}}
\label{ch:inleiding}

\section{\IfLanguageName{dutch}{Probleemstelling}{Problem Statement}}
\label{sec:probleemstelling}
Dit onderwerp werd voorgesteld door Delaware. De doelgroep is dus Delaware.

Uit je probleemstelling moet duidelijk zijn dat je onderzoek een meerwaarde heeft voor een concrete doelgroep. De doelgroep moet goed gedefinieerd en afgelijnd zijn. Doelgroepen als ``bedrijven,'' ``KMO's,'' systeembeheerders, enz.~zijn nog te vaag. Als je een lijstje kan maken van de personen/organisaties die een meerwaarde zullen vinden in deze bachelorproef (dit is eigenlijk je steekproefkader), dan is dat een indicatie dat de doelgroep goed gedefinieerd is. Dit kan een enkel bedrijf zijn of zelfs één persoon (je co-promotor/opdrachtgever).

\section{\IfLanguageName{dutch}{Onderzoeksvraag}{Research question}}
\label{sec:onderzoeksvraag}
De algmene onderzoeksvraag is: "Hoe microserivce integration patterns een order-to-cash proces in SAP kan beïnvloeden?". De algemene vraag delen we op in volgende puntjes:
\begin{itemize}
  \item Wat zijn microservices?
  \item Hoe kan er overgeschakeld worden naar een microservice architectuur?
  \item Welke aanpassingen kunnen of moeten er gebeuren aan de architectuur om de microservices te laten werken?
  \item Hoe zal de communicatie tussen de verschillende microservices werken?
  \item Hoe zit een order-to-cash (OTC) proces er uit?
  \item Welke business requirements heeft een order-to-cash proces?
  \item Welke invloed heeft de microservice architectuur op de performance van een order-to-cash proces?
\end{itemize}
Dit zal een theoretische studie zijn.

\section{\IfLanguageName{dutch}{Onderzoeksdoelstelling}{Research objective}}
\label{sec:onderzoeksdoelstelling}
Het doel van deze paper is het onderzoeken van de effecten van een microservices architectuur op de architectuur van een OTC in SAP. Het doel houdt ook in dat we aan de hand van een zes-stappen plan, theoretisch, gaan kijken hoe een OTC verandert door microservices.  

\section{\IfLanguageName{dutch}{Opzet van deze bachelorproef}{Structure of this bachelor thesis}}
\label{sec:opzet-bachelorproef}

% Het is gebruikelijk aan het einde van de inleiding een overzicht te
% geven van de opbouw van de rest van de tekst. Deze sectie bevat al een aanzet
% die je kan aanvullen/aanpassen in functie van je eigen tekst.

De rest van deze bachelorproef is als volgt opgebouwd:

In Hoofdstuk~\ref{ch:stand-van-zaken} wordt een overzicht gegeven van de stand van zaken binnen het onderzoeksdomein, op basis van een literatuurstudie. Hierin zal er meer uitleg gegeven worden over microservices en het order-to-cash proces binnen SAP.

In Hoofdstuk~\ref{ch:methodologie} wordt de methodologie toegelicht en worden de gebruikte onderzoekstechnieken besproken om een antwoord te kunnen formuleren op de onderzoeksvragen. Hier zal het onderzoek worden uitgevoerd over hoe microservices het order-to-cash proces zullen beïnvloeden.

% TODO: Vul hier aan voor je eigen hoofstukken, één of twee zinnen per hoofdstuk

In Hoofdstuk~\ref{ch:conclusie}, tenslotte, wordt de conclusie gegeven en een antwoord geformuleerd op de onderzoeksvragen. Daarbij wordt ook een aanzet gegeven voor toekomstig onderzoek binnen dit domein.
