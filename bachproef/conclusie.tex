%%=============================================================================
%% Conclusie
%%=============================================================================

\chapter{Conclusie}
\label{ch:conclusie}

% TODO: Trek een duidelijke conclusie, in de vorm van een antwoord op de
% onderzoeksvra(a)g(en). Wat was jouw bijdrage aan het onderzoeksdomein en
% hoe biedt dit meerwaarde aan het vakgebied/doelgroep? 
% Reflecteer kritisch over het resultaat. In Engelse teksten wordt deze sectie
% ``Discussion'' genoemd. Had je deze uitkomst verwacht? Zijn er zaken die nog
% niet duidelijk zijn?
% Heeft het onderzoek geleid tot nieuwe vragen die uitnodigen tot verder 
%onderzoek?

De conclusie van deze theoretische studie is dat aan de hand van microservices, het order-to-cash proces geautomatiseerd kan worden. 

Microservice kan in meerdere contexten voor komen. Binnen elke context zal er meer verkaring nodig zijn.

Authorisatie en authenticatie kan toe gepast worden op verscheidene manieren. Naargelang de noden van de applicatie kan de wijze van deze verschillen. 

Bescherming kan op verschillende manieren toegepast worden. Hier moet er aan grondige research gedaan worden om te beslissen welke manier de beste is. 

Microservice is een architectuur en ideologie waarin logica en de requirements, die te vinden zijn bij een monolithic, terugkomen. Dit heeft invloed op de overschakeling naar microservices. Deze manier van implementatie moet aangepast worden bij het gebruik van microservices. Dat kan een obstakel zijn. Er wordt een andere manier van denken en organiseren gevraagd binnen een de IT-afdeling. De teams worden hervormd. Een team bevat niet meer de kennis van de volledige architectuur., enkel over de microservice waaraan zij werken. 

Een ander punt dat aan bodt komt, is de veranderingen die gebeuren. Die worden in volgende opsomming weergegeven:
\begin{itemize}
	\item Heeft deze applicatie nood aan een microservice architectuur?
	\item De databank structuur moet herbekeken worden.
	\item Het volledige proces herbekijken.
	\item De manier van communiceren tussen de verschillende onderdelen moet worden onderzocht.
	\item Welke manier van authenticatie en authorisatie gaat men gebruiken?
	\item Welke manier van bescherming kan er toegepast worden?
	\item Moeten we de teams herstructureren?
\end{itemize}

De veranderingen die werden toegepast aan het OTC-proces zijn gelijklopend met de opsomming hierboven. 
Door de uitwerking, is het duidelijk geworden hoe microservices meerdere keren gebruikt kunnen worden. Als men kijkt naar de microservice 'klantgegevens ophalen', ziet men dat deze meermaals gebruikt wordt binnen de architectuur. 

Ten slotte kan er geconcludeerd worden dat:
\begin{itemize}
	\item Microservices een brede term is en kan voorkomen in verschillende contexten.
	\item Een microservice architectuur niet altijd nodig is.
	\item Microservices kunnen een OTC proces automatiseren.
	\item Microservices zorgen ervoor dat elk onderdeel van het proces apart kan functioneren.
\end{itemize}

Zelf heb ik ondervonden dat het aanpassen van een monolithic niet eenvoudig is. De monolithic architectuur zit ingebakken in het informatica landschap. De manier van omgang met microservices, kan vergeleken worden met die van Agile. De Agile methode moest in eerste instatie zijn toegevoegde waarde op een project bewijzen. Dan pas zijn projectmanagers deze ook gaan gebruiken. Microservices zal eerst zijn voordeel moeten bewijzen ten opzichte van monolithic om een statement te kunnen maken.



